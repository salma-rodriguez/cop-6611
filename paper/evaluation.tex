\section{Evaluation}

We will use two approaches for our evaluation. The first
is a practical approach, where we will use one of the
failure types introduced in section~\ref{approach} to evaluate
the design for reliability. The failure type we will use is
induced power failure, where power failure will purposely be
induced simply with one or two replicas in the caching system.

The second approach is a theoretical approach. Reliability is
quantified as mean time to failure (MTTF). The time-dependent
mean time between failure (MTBF) is given by the equation:

\begin{equation}
\theta = \frac{T}{N}
\end{equation}

where MTBF converges to MTTF as the number of failing devices
approach the total number of devices in the system.
In this equation,
$T$ is the total time, aggregated for all the nodes, so that if
there are $n$ nodes in the system and each performs I/O-bound
activities on its locally attached SSD for time $t$, then the
total time, $T$ is \[n\times{t}\]. In essence, the
higher the number of replicas in the system, the lower the
number of failing caches in proportion to $T$, and this is what
we want to prove through statistical methods.

Permitted that we have the hardware and time, we will use our
statistical parameter

\begin{equation}
	\lambda = \frac{1}{\theta}
\end{equation}

against time
to graphically determine the reliability curve for real life
workloads. This curve is given by the following equation:

\begin{equation}
	f(t) = \lambda e^{-\lambda t}
\end{equation}

MTTF can be accurately determined by integrating (4) over time
as follows:

\begin{equation}
	\frac{\int_0^\infty tf(t)dt}{\int_0^\infty f(t)d(t)}
\end{equation}

This, of course, then reduces to:

\begin{equation}
	\int_0^\infty tf(t)dt
\end{equation}

since $f(t)$ approaches 1 as $t$ approaches $\infty$. Finally,
we can integrate $f(t)$ to generate a cummulative distribution
function (CDF), as follows:

\begin{equation}
	\int_{t_0}^{t_1} \lambda e^{-\lambda t}dt
\end{equation}

but when $t_0$ is zero, this integral simply reduces to:

\begin{equation}
	F(t) = 1 - e^{-\lambda t_1}
\end{equation}

which we will use to calculate the probability of failure for
$t < t_1$. The probability of failure for $t_1 > 1$ is given by

\begin{equation}
	R(t) = 1 - F(t)
\end{equation}

\label{evaluation}
