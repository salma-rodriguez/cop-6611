\section{Implementation}
%
\begin{frame}
    \frametitle{Framework}
    \begin{itemize}
	\item Modify an existing block-level caching solution
	\item Use block request mapping and redirection mechanism
	\item Add components for cooperative caching and replication
	\item Add components to handle failure and recovery
    \end{itemize}
\end{frame}
\begin{frame}
    \frametitle{Failure \& Recovery}
    \begin{itemize}
	\item DM Cache keeps the source device in the bi bdev
	    of each block I/O, with block addresses being mapped
	    to the cache device
	\item Need to partition the physical cache and set up
	    each logical cache
	\item Each logical cache maps block addresses from the
	    source device independently
	\item Already have source device. Need to flush data when
	    replicator is not available
	\item Server maintains redundancy through RAID setup but
	    may still keep data when not available and flush later
    \end{itemize}
\end{frame}
\begin{frame}
    \frametitle{Policy for Failure Handling}
    Assumption: not all replicas are consistent. \\
    \bf Algorithm 1 \rm Flushing the data on failure. \\
    \begin{algorithmic}[1]
	\Procedure{Do Failover}{}
	    \State $D\gets$ server disk
	    \State $C\gets$ replicator cache
	    \State $T\gets$ time specified by user
	    \While{true}
		\If{C and D available}
		    sleep for $T$ seconds
		    \Else \Comment{One of C or D are not available}
			\State $V\gets$ list of bios (latest copy)
			\If{D is not available}
			    \State keep $V$ until server disk is available
			    \State flush $V$ back to source device
			\Else \Comment{C is not available}
			    \State flush $V$ back to source device
		    \EndIf
		\EndIf
	    \EndWhile
	\EndProcedure
    \end{algorithmic}   
\end{frame}
