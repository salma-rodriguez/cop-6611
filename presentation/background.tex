\section{Background}
%
\begin{frame}
    \frametitle{Device Mapper Cache}
    \begin{block}{What is it?}
	A generic block-level caching mechanism
	for storage networks
    \end{block}
    \vspace{2ex}
    \begin{block}{What does it do?}
	Cache popular data locally to reduce
	network and storage server latency
    \end{block}
    \vspace{2ex}
    \begin{block}{How does it work?}
	Built upon the Linux kernel device mapper, which
	maintains the mapping between a source device, and
	a cache device.
    \end{block}
\end{frame}
\begin{frame}
    \frametitle{Device Mapper Cache}
    \begin{tabular}{m{0.465\linewidth}m{0.465\linewidth}}
	%\hline
	DM Cache takes advantage of spatial and temporal
	locality by caching data locally and using LRU for
	cache replacement. &
	\begin{figure}
	    %\caption{Device Mapper Layout}
	    \centering \includegraphics[scale=.23]{current.png}
	    \label{fig:dm}
	\end{figure} \\
	%\hline
    \end{tabular}
    \begin{block}{Device Mapper Cache}
	\begin{itemize}
	    \item Distributed shared storage systems (SAN, iSCSI, AoE, \&c.)
		support better scalibility by using block-level caching
		on the client side
	    \item Fast mass storage devices, like Solid State Drives (SSDs) 
		are excellent candidates for cache devices
	    \item While memory can
		still support faster I/O, SSD caches provide greater capacity
	\end{itemize}
    \end{block}
\end{frame}
\begin{frame}
    \frametitle{Replication}
    One-to-many distribution of data from a \textit{source}
    to several \textit{targets} to hold replicas of the data
    \begin{block}{Pessimistic Replication}
	\begin{itemize}
	    \item synchronous
	    \item consistent, but not optimal: system may incur
		performance bottleneck from unpredictable network
		and storage latency
	    \item low availability: replicator will block until data is
		fully propagated to all targets
	\end{itemize}
    \end{block}
    \begin{block}{Optimistic Replication}
	\begin{itemize}
	    \item asynchronous
	    \item high availability, low latency, not consistent
	    \item less reliable than pessimistic replication
	\end{itemize}
    \end{block}
\end{frame}
\begin{frame}
    \frametitle{Cooperative Caching}
    \begin{block}{Idea}
	Collaboration across a network in order to saturate
	the available space for caching
    \end{block}
    \begin{block}{Implementation}
	\begin{itemize}
	    \item IP-based interface for devices to communicate (i.e., iSCSI)
	    \item iSCSI works at the block layer. Only need a cache
		partition for replicas
	\end{itemize}
    \end{block}
\end{frame}
