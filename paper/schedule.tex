\section{Schedule}

Current Progress \\

We have designed ways to replicate data and cooperate
among clients. Implementation has begun, and we are still
working on our prototype. Our current implementation
includes ways to submit block I/Os to more than one
device, and a working peer to peer communication protocol
using TCP.

Jorge is working on cache partitioning and replication.
He is figuring out a way to submit block I/Os as a bio
request to more than one device. He will also implement
the partitioning algorithm that will be used to store
replicas for peer caches.

Salma is working on serialization of block I/Os
which will be sent through TCP sockets across the
network, and reconstructed by the receiving clients.
Ways to deserialized block I/O data and submit these to
the block layer queue for writing to the block device
are still being implemented. \\

\setlength{\parindent}{0cm}
November 15 - 21 \\
\setlength{\parindent}{1 em}

Jorge will continue working on the replication policies.
Salma will continue working on cooperative caching.
Jorge needs to find a way to partition the cache, so that
a portion of it is reserved for peer clients. He also
needs to explore ways to pin the pages of peer caches, so
that the data is not accidentally erased while unwritten
dirty data is still hanging around, waiting to be written
back to the storage server. Salma needs to find a way
to serialize block I/O data structures, write these to a
socket file discriptor, with the data being read, and
deserialized from the target client. Salma will implement
a way to construct the block I/O request and submit for
writing to the device. The mechanism will include an
acknowledgement with the sector number where the block of
data was stored. \\

\setlength{\parindent}{0cm}
November 21 - 28 \\
\setlength{\parindent}{1 em}

A table with client metadata still needs to be
implemented, so that information about replicas is kept in
volatile memory and also reflected on persistent storage
periodically. The implementation of replication will also
include a way to mark the data as invalid in the replica
targets when replicas have been written back to disk
and are no longer needed. Salma will work on creating and
persisting the metadata, and Jorge will implement a
mechanism for invalidating replicas after every writeback.
Both Jorge and Salma will be meeting with the professor
and classmates for feedback and suggestions as the deadline
approaches. \\

\setlength{\parindent}{0cm}
November 28 - 5 \\
\setlength{\parindent}{1 em}

Use feedback and suggestions to finish whatever is missing
before December 1st, and from December 1st to the 5th,
both Jorge and Salma will be preparing the presentation and
the classroom demonstration. Both Jorge and Salma will add
figures to their slides and add bibliographical refrences
to improve the quality over their first presentation. \\

\setlength{\parindent}{0cm}
December 5 - 12 \\
\setlength{\parindent}{1 em}

Add whatever is missing to the paper from December 5 to 9.
We realize that we need figures in our paper. These will be
added along the way. Both Jorge and Salma will polish the
paper before turning it in on December 12.

\label{schedule}
